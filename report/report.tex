\documentclass[11pt]{article}
\usepackage{graphicx}
\usepackage[margin=2.5cm]{geometry}
\usepackage{tikz}
\usepackage{indentfirst}
\usepackage{tabularx}
\usepackage{listingsutf8}
\usepackage{color}
\usepackage{hyperref}
\usepackage{float}
\usepackage[portuguese]{babel}

\graphicspath{{./images/}}

\def\checkmark{\tikz\fill[scale=0.4](0,.35) -- (.25,0) -- (1,.7) -- (.25,.15) -- cycle;} 
\setlength{\parskip}{0.5em}

\renewcommand{\lstlistingname}{Código}
\renewcommand{\lstlistlistingname}{Pedaços de Código}

\definecolor{dkgreen}{rgb}{0,0.6,0}
\definecolor{gray}{rgb}{0.5,0.5,0.5}
\definecolor{mauve}{rgb}{0.58,0,0.82}

\hypersetup{
	colorlinks=false,
	linktoc=all,
	hidelinks,
}

\lstset{
	belowcaptionskip=1\baselineskip,
	captionpos=b,
	frame=tb,
	language=Java,
	aboveskip=3mm,
	belowskip=3mm,
	showstringspaces=false,
	columns=flexible,
	basicstyle={\small\ttfamily},
	numbers=none,
	numberstyle=\tiny\color{gray},
	keywordstyle=\color{blue},
	commentstyle=\color{dkgreen},
	stringstyle=\color{mauve},
	breaklines=true,
	breakatwhitespace=true,
	tabsize=3,
	inputencoding=utf8,
	extendedchars=true,
	literate={á}{{\'a}}1 {ã}{{\~a}}1 {à}{{\`a} }1 {Ã}{{\~A}}1 {ó}{{\'o}}1 {Ó}{{\'O}}1 {Í}{{\'I}}1 {í}{{\'i}}1 {é}{{\'e}}1 {ç}{{\c{c}}}1 {Ç}{{\c{C}}}1 {ú}{{\'u}}1 {õ}{{\~o}}1
}

\begin{document}
	\begin{titlepage}
		\begin{center}
			\includegraphics[width=0.6\textwidth]{logo-isec}
			
			\vspace*{\fill}
			
			\includegraphics[width=0.4\textwidth]{icon-weather}
			
			\Huge
			\textbf{Weather}
			
			\huge
			Relatório Técnico
			
			\vspace{2cm}
			
			\Large
			\textbf{
				Ângelo Paiva, 2019129023 \\
				Jan Frank, 2017009793 \\
				Pedro Henriques, 2019129770
			}
			
			\vfill
			\vspace*{\fill}
			
			\normalsize
			Arquiteturas Móveis \\
			Licenciatura de Engenharia Informática, Ramo de Desenvolvimento de Aplicações \\
			22 de janeiro de 2022		
		\end{center}
	\end{titlepage}

	\tableofcontents
	\pagebreak
	
	\large
	\section{Introdução}
	\normalsize
	
	Neste trabalho foi projetada uma aplicação para consultar metereologia. Esta aplicação foi escrita em Dart com recurso a Flutter.
	
	Este relatório irá servir para explicar a implementação e a razão por trás das decisões que foram tomadas.
	
	
	\large
	\section{API Escolhida}
	\normalsize
	
	A API que escolhemos para a realização deste trabalho foi \color{blue}\underline{\href{https://openweathermap.org/}{OpenWeatherMap}}\color{black}. Achamos que esta fornece bastante informação e, ao ler reviews online, chegámos também à conclusão que é das que tem mais precisão em relação às previsões.
	
	Para além disto, esta tem também um elevado limite de pedidos para a versão gratuita, podendo fazer até 60 pedidos por segundo ou 1,000,000 de pedidos por mês.
	
	Apesar de fornecer informação para minuto, decidimos ignorar porque não era detalhada o suficiente para ser útil para o utilizador normal.
	
	
	\large
	\section{Shared Preferences}
	\normalsize
	
	A última atualização feita é sempre guardada offline com recurso a shared preferences. Nesta temos dois campos, um para guardar o nome da cidade onde o pedido foi feito chamado \textbf{city} e outro onde guardamos toda a informação recebida da API chamado \textbf{weatherData}.
	
	Para guardar a informação recebida da API nas shared preferences foi necessário codificar o JSON para String.
	
	
	\large
	\section{Internacionalização}
	\normalsize
	
	A nossa aplicação tem suporte para inglês e português com recurso ao plugin intl.
	
	
	\large
	\section{Interface do Utilizador}
	\normalsize
	
	A nossa interface foi inspirada \color{blue}\underline{\href{https://search.muz.li/OThiNGFlMTE2}{nesta interface}}\color{black}. Apesar de termos inicialmente baseado nessa, consideramos que a nossa interface se tornou única e com a sua própria personalidade com o decorrer do projeto.
	
	O objetivo era ter uma interface minimalista mas que não falhasse a apresentar informação ao utilizador.


	\large
	\section{Suporte para iOS}
	\normalsize
	
	Como tínhamos membros com máquina da Apple, a aplicação foi testada em iOS e funciona corretamente.
	

	\large
	\section{Conclusão}
	\normalsize
	
	Para além de ter sido um verdadeiro desafio, este trabalho foi uma excelente oportunidade para conhecer melhor \textbf{Flutter} e todas as vantagens que nos proporciona.
	

	\large
	\section{Anexos}
	\normalsize
	
	\listoffigures
	\lstlistoflistings
\end{document}